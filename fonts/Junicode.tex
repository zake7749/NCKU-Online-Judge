%&program=xelatex
%&encoding=UTF-8 Unicode

\documentclass[a4paper]{article}

\usepackage{fontspec}

\usepackage{microtype}

\setromanfont{Junicode}

\newICUfeature{StyleSet}{1}{+ss01}
\newICUfeature{StyleSet}{insular}{+ss02,-liga}
\newICUfeature{StyleSet}{highline}{+ss04}
\newICUfeature{StyleSet}{medline}{+ss05}
\newICUfeature{StyleSet}{enlarged}{+ss06}
\newICUfeature{StyleSet}{underdot}{+ss07}
\newICUfeature{StyleSet}{altyogh}{+ss08}
\newICUfeature{StyleSet}{altpua}{+ss09}
\newICUfeature{StyleSet}{althook}{+ss14}
\newICUfeature{StyleSet}{altogonek}{+ss15}
\newICUfeature{StyleSet}{oldpunct}{+ss18}
\newICUfeature{MirrorRunes}{on}{+rtlm}
\newICUfeature{LigType}{disc}{+dlig}
\newICUfeature{LigType}{hist}{+hlig}
\newICUfeature{LongS}{on}{+hist,+fina}
\newICUfeature{IPAMode}{on}{+mgrk,-liga}
\newICUfeature{Compose}{off}{-ccmp}
\newICUfeature{Contextual}{on}{+calt}
\newICUfeature{Swash}{on}{+swsh}
\newICUfeature{Fractions}{on}{+frac}
\newICUfeature{Superscripts}{on}{+sups}
\newICUfeature{Subscripts}{on}{+subs}
\newcommand{\sampletext}{Lorem ipsum dolor sit amet, consectetur adipisicing elit, sed do eiusmod tempor incididunt ut labore et dolore magna aliqua. 12345 \fontspec[Numbers=OldStyle]{Junicode}12345}
\newcommand{\sctext}{Cum multa divinitus, pontifices, a
ma\-ioribus nos\-tris in\-venta atque in\-sti\-tuta sunt}

\frenchspacing

\begin{document}
\large
\section*{Junicode}

\fontspec[Contextual=on]{Junicode}Version 0.7.2\\

\noindent The Junicode font is designed to
meet the needs of medieval scholars; however, it has a large enough
character set to be useful to the general user. It comes in Regular,
Italic, Bold and Bold Italic faces. The Regular face has the fullest
character set, and is richest in OpenType features.

Both the selection and design of the characters in Junicode reflect
the needs of medievalists.  However, many persons writing in ancient
and modern languages have found the font useful. If you wish to see
better support for any language, please leave a request at the
Junicode project page (http://sourceforge.net/projects/junicode).

Junicode implements most of the recommendation of the Medieval Unicode
Font Initiative (version 3.0).  Look for special MUFI characters
(those not in the Unicode standard) in the Private Use Area (U+E000
and above). Download the complete recommendation at
http://www.mufi.info/.

Junicode is licensed under the SIL Open Font License: for the full
text, go to
http://scripts.sil.org/OFL. Briefly: You may use Junicode in any
kind of publication, print or electronic, without fee or
restriction. You may modify the font for your own use. You may
distribute your modified version in accordance with the terms of the
license.

\pagebreak
\subsection*{OpenType Features}

{\itshape Only OpenType-aware applications can make use of OpenType
  features.  Among these are Adobe InDesign, Mellel, and (to a limited
  extent) Microsoft Word. XeTeX, a typesetting program built on top of
  TeX, has especially good support. The following are standard
  OpenType features (not all available in all faces). For OpenType
  features especially for medievalists, see the next section.}\\

\noindent Like many old-style fonts, Junicode contains several f-ligatures
(first flight offer office afflict fjord). It also has a number of
other standard ligatures, e.g. thrift fifty afraid für fördern.  It
also has long-s ligatures (aſſert ſtart ſlick omiſſion and more). Most
OpenType-aware applications will use these by default. You can disable
them by turning off “Standard Ligatures” (liga). {\itshape All
  faces.}\\

\noindent If “Contextual Alternates” (calt) and “Horizontal Kerning” (kern)
are on (as they should be by default), Junicode will avoid collisions
between f and vowels with diacritics, e.g.  fêler fíf fŭl.\\

\noindent For circled numbers and letters, use “Discretionary Ligatures” (dlig):\\
\hspace*{10pt}[1] {\fontspec[LigType=disc]{Junicode}= [1]}\\
\hspace*{10pt}[A] {\fontspec[LigType=disc]{Junicode}= [A]}\\
\hspace*{10pt}[a] {\fontspec[LigType=disc]{Junicode}= [a]}\\
\hspace*{10pt}[[1]] {\fontspec[LigType=disc]{Junicode}= [[1]]}\\
\hspace*{10pt}<1> {\fontspec[LigType=disc]{Junicode}= <1>}\\
The same feature also gives you connected Roman numbers
{\fontspec[LigType=disc]{Junicode}(I II III IV V VI VII VIII IX X XI
  XII), and fancy ligatures, e.g. act star track bitten attract.}
{\itshape Regular face only.}\\

\noindent With “Glyph Composition/Decomposition” (ccmp), a base
character followed by one or more combining diacritical marks is
replaced with a precomposed character when that would look different
from the character + diacritic sequence: for example A + U+301 makes Á,
where a special upper-case form of the diacritic is used. {\itshape
  All faces, depending on the availability of composed characters and
  combining diacritics.}\\

\noindent Where no precomposed character is available, combining marks should
still be correctly positioned, and marks can be “stacked” via “Mark
to Base” (mark) and “Mark to Mark” (mkmk): ŏ́ (o + U+306 + U+301);
ī̆ (i + U+304 + U+306).  The dot of an i or j followed by a diacritic
will generally be removed: i̽. {\itshape All faces; anchors less
  plentiful in bold and italic faces than in regular; diacritic
  stacking not available in bold italic.}\\

\noindent Use “Small Caps” (smcp) to change lower-case letters to
small caps; add “Caps to Small Caps” (c2sc) for text entirely in small
caps. {\scshape Junicode has true small caps rather than scaled
  capitals.} Special small cap versions of common combining diacritics
are available, and these should be positioned correctly relative to
the base characters: {\scshape äçé}. {\itshape Regular face
  only.}\\

\noindent You have a choice of either standard “lining” figures or old-style
figures, selected by “Old-Style Numbers” (onum): 0123456789
{\fontspec[Numbers=OldStyle]{Junicode}0123456789.} {\itshape All
  faces.}\\

\noindent Superscript numbers are rendered with “Superscripts” (sups):
{\fontspec[Superscripts=on]{Junicode} 0123456789}.  Subscript numbers
are rendered with “Subscripts” (subs):
{\fontspec[Subscripts=on]{Junicode} 0123456789}.  {\itshape Regular
  only.}\\

\noindent A sequence of number + slash + number is rendered by a fraction if the
fraction has a Unicode encoding and “Fractions” (frac) is on:
{\fontspec[Fractions=on]{Junicode} 1/2 1/4 2/3 3/4}. {\itshape All
  faces, but fullest in regular and bold.}\\

\noindent The design of a few Junicode characters has changed since the font was
introduced. The original designs, if you prefer them, will always be
available via “Style Set 9” (ss09). Currently there are just a few
such alternates: {\fontspec[StyleSet=altpua]{Junicode} ꝺ} for ꝺ,
{\fontspec[StyleSet=altpua]{Junicode} T} for T,
{\scshape{\fontspec[StyleSet=altpua]{Junicode} t} for t}.\\

\noindent{\addfontfeature{StyleSet=oldpunct}Old books generally set
extra space before the heavier punctuation marks (; : ! ?);
they also leave extra space inside quotation marks and
parentheses (e.g. “here”). For a similar effect, use Stylistic Set 18 (ss18). Make sure
that Contextual Alternates are also on so that Junicode can correct
the spacing in certain environments (but you will have to kern the English plural
possessive apostrophe manually).}\\

\noindent For XeLaTeX users who use the Microtype package for
character protrusion, a
configuration file (mt-Junicode.cfg) is provided for Junicode. Users
of XeLaTeX will need Microtype version 2.5 (currently beta). The
configuration file will work only with XeLaTeX, though it can probably be made
to work with LuaTeX by commenting out the last five lines of the
{\textbackslash}DeclareCharacterInheritance command.

\pagebreak
\subsection*{Notes on Junicode and MUFI}

The MUFI specification defines a great many characters of interest to
medievalists; the current version of Junicode contains most of
these. While many MUFI characters have Unicode encodings, many others
have MUFI-recommended encodings in the “Private Use Area” (PUA) of the
Unicode standard—that is, a range of code points not assigned to
Unicode characters and available for use in fonts for specialized
purposes. Use of the PUA allows MUFI to include many characters that
are not part of the Unicode standard.

There are risks with this approach. First, use of the PUA is
deprecated by Adobe and Microsoft (major players in fonts and type rendering),
and it is uncertain whether applications will continue to support it
indefinitely. Second, and probably more important, MUFI characters are
regularly accepted by the Unicode Consortium, whereupon they lose their
PUA encodings and receive Unicode encodings—breaking any
application that uses them.

To minimize these risks, the MUFI specification strongly recommends “that PUA
characters should be encoded with mark-up or entities, and that PUA characters should be used for the final display only, whether on screen or in print.” An alternative to
entities is the use of OpenType features. If you are using an OpenType-aware
application (e.g. XeTeX, InDesign, Mellel, and to a limited extent MS Word),
many or all of the OpenType features of Junicode can help you avoid using PUA
characters directly.\\

\noindent {\bfseries Characters with diacritics.}
Both Unicode and MUFI contain large numbers of characters with diacritics.
Make it a habit never to use these “precomposed” characters directly; rather
use the “plain” character followed by a character from the Unicode “Combining
Diacritics” range. (This works with Word for Windows when Uniscribe is
enabled, and also with other OpenType-aware applications.) In almost all cases
the application will either substitute the correct precomposed character or
position the diacritic correctly. For characters with more than one diacritic,
follow these rules: when diacritics are stacked vertically, insert the one
closest to the base character first; when diacritics are arranged horizontally,
insert the leftmost one first. Examples: a + macron + acute = ā́; o + dot +
acute = . Remember that Unicode has both spacing and combining diacritics;
only the combining diacritics will work correctly. If any combination fails to
work for you, please leave a bug report at the Junicode website.\\

\noindent {\bfseries Small caps.}
Make it a hard-and-fast rule {\itshape never} to insert any small cap character
into your documents. The encoding of small caps is inherently unstable and
non-portable. Early versions of MUFI recommend using small cap-like characters from the
Unicode phonetic ranges, but this would be an error with many fonts, including
Junicode, which size phonetic “small caps” to harmonize with lower-case
characters, whereas true small caps are somewhat larger. Always use the
small caps command provided by the application you are using. If the
application is able, it will use Junicode’s true small caps.

You may use the “small caps” in the phonetic ranges to set IPA text. The
“small cap” ʀ is also recommended for setting transliterations of early Norse
runic texts.\\

\noindent {\bfseries Nordic letter-shapes.}
The default shape of ð and þ in Junicode is English: this is unusual in
modern fonts. For the shapes used in Icelandic, specify the Icelandic
language, if your application has good language support, or select
“Style Set 1” (ss01): {\fontspec[Language=Icelandic]{Junicode} ðþ}.\\

\noindent {\bfseries Insular letter-shapes.}  Insular letter-forms have
recently been accepted by Unicode, and therefore their encodings have
changed. For Junicode, use “Style Set 2” (ss02) for insular
letter-forms if your application supports it:
{\fontspec[StyleSet=insular]{Junicode} abcdefg.} Turn off “Standard
Ligatures” (liga) for best results.\\

\noindent {\bfseries Old English and Old Icelandic typography.}  When
Old English or Old Icelandic is set with Junicode, some letter
combinations can produce unattractive collisions. To avoid this, make
sure that “Contextual Alternates” (calt) and “Standard Ligatures”
(liga) are on (as they should be by default): hæfð hæfþ fūl nīð.\\

\noindent {\bfseries Enlarged minuscules.}  In Junicode, “Style Set 6”
(ss06) produces enlarged minuscules, thus:
{\fontspec[StyleSet=enlarged]{Junicode} abcdefg.} Since the underlying
text remains unchanged, enlarged text can be searched like normal
text.\\

\noindent {\bfseries Overlined characters.}  The MUFI specification
envisions a font-based mechanism for producing text with
overlines. Probably this will not be practical in the near future;
rather, use your application’s line-drawing facilities to produce text
with overlines. For Junicode, roman numbers are an exception. Use
“Style Set 4” (ss04) for roman numbers with high overline
({\fontspec[StyleSet=highline]{Junicode} viii XCXV}) and “Style Set 5”
(ss05) for lower-case roman numbers with medium-high overline
({\fontspec[StyleSet=medline]{Junicode} viii dclx}).\\

\noindent {\bfseries Letters with hook above.} The Unicode standard
contains several precomposed characters with combining hook above in
the Latin Extended Additional range (e.g. ẢỎ). These are used
automatically when a vowel is followed by the diacritic
U+0309. However, MUFI contains a series of precomposed characters in
which the hook differs in shape and position. Use “Style Set 14”
(ss14) for the MUFI characters (e.g.
\fontspec[StyleSet=althook]{Junicode}ẢỎ).\\

\noindent {\bfseries Letters with flourishes.}
For letters with flourishes (sometimes used for setting Middle English
texts), use “Swash” (swsh):
{\fontspec[Swash=on]{Junicode}c d f g k n r}.\\

\noindent {\bfseries Ligatures.}  Nearly all of MUFI’s ligatures are
accessible via “Historical Ligatures” (hlig).
{\fontspec[LigType=hist]{Junicode}Even if you are not a medievalist,
  you may still be amused by the strange effects you can achieve by
  turning on this feature: egg track caught fan sock book save aardvark
  chaos AA AO
  AU AV.}\\

\noindent {\bfseries Deleted text.}
In medieval manuscripts, text is often deleted by placing a dot under each
letter. Both Unicode and MUFI define many characters with dots below:
{\fontspec[StyleSet=underdot]{Junicode} if possible, you should avoid
hard-coding these and instead use} “Style Set 7” (ss07).\\

\noindent {\bfseries Alternate yogh.}
For Middle English, always use the yogh at U+021C and U+021D (Ȝȝ).
Unicode also has an alternative yogh, which in Junicode has a
flat top. If you prefer this, leave the underlying text the same and
specify “Style Set 8” (ss08):
{\fontspec[StyleSet=altyogh]{Junicode} Ȝȝ}.\\

\noindent {\bfseries Deprecated characters.} A number of characters
which were encoded in the Private Use Area in MUFI versions 1 and 2
have been adopted by Unicode and now have different code points. In
Junicode these characters remain at their old locations, but are
marked with a small “x” to remind users to migrate to the newer code
points (e.g. ). The file “replacements” contains a list of these
deprecated code points with their replacements; use this to update
your documents. If you are unable to change the encoding of an older
document but you can use OpenType features, turn on “Style Set 3”
(ss03); this will automatically substitute newer for older code
points.\\

\noindent {\bfseries E caudata.} Medieval Latin texts often use an
          {\itshape e} with tail, called {\itshape e caudata}; this
          represents Latin {\itshape ae} or {\itshape oe}. Polish,
          Lithuanian, and several other languages also use this
          letter. While in modern editions of medieval texts the
          {\itshape cauda} (or in Polish, the {\itshape ogonek}) is
          often attached to the very bottom of the letter, in modern
          Polish and Lithuanian printing it is attached to the end of
          the bottom stroke: Polish ę, medieval Latin
          {\addfontfeatures{StyleSet=altogonek}ę}. The modern Polish
          version of the letter is acceptable for medieval Latin;
          however, if you prefer a centered {\itshape cauda}, use
          “Style Set 15” (ss15).\\

\noindent {\bfseries Mirrored runes.} In the regular face Junicode
contains mirrored versions of runes. To access these, use
Right-to-Left Mirroring (rtlm): {\addfontfeatures{MirrorRunes=on} ᚾᚪᛒᛋᚫᚾᚩᚱᚻ.}

\pagebreak
\subsection*{Old and Middle English}

Wē æthrynon mid ūrum ārum þā ȳðan þæs dēopan wǣles; wē
ġesāwon ēac þā muntas ymbe þǣre sealtan sǣ strande, and wē mid
āðēnedum hræġle and ġesundfullum windum þǣr ġewīcodon on þām
ġemǣrum þǣre fæġerestan þēode. Þā ȳðan ġetācniað þisne dēopan
cræft, and þā muntas ġetācniað ēac þā miċelnyssa þisses cræftes.\\


\noindent{\small\itshape Apply the OpenType feature ss02 (Style Set 2)
for insular letter-forms.}\\[1ex]
\fontspec[StyleSet=insular]{Junicode}
Her cynewulf benam sigebryht his rices \& westseaxna wiotan for
un\-ryht\-um dædum buton hamtunscire \& he hæfde þa oþ he ofslog
þone aldorman þe him lengest wunode \& hiene þa cynewulf on
andred adræfde \& he þær wunade oþ þæt hine an swan ofstang
æt pryfetesflodan \& he wræc þone aldorman cumbran \& se cynewulf
oft miclum gefeohtum feaht uuiþ bretwalum.\\

\fontspec{Junicode}
\noindent SIÞEN þe sege and þe assaut watz sesed at Troye,\\
Þe borȝ brittened and brent to brondez and askez,\\
Þe tulk þat þe trammes of tresoun þer wroȝt\\
Watz tried for his tricherie, þe trewest on erthe:\\
Hit watz Ennias þe athel, and his highe kynde,\\
Þat siþen depreced prouinces, and patrounes bicome\\
Welneȝe of al þe wele in þe west iles.

\subsection*{Old Icelandic}

\fontspec[Language=Icelandic]{Junicode}
{\small\itshape For Nordic shapes of þ and ð, specify the Icelandic
language, if your application has good language support; or apply the OpenType
ss01 (Style Set 1) feature.}\\[1ex]
Um haustit sendi Mǫrðr Valgarðsson orð at Gunnarr myndi vera einn heimi, en
lið alt myndi vera niðri í eyjum at lúka heyverkum. Riðu þeir Gizurr Hvíti ok
Geirr Goði austr yfir ár, þegar þeir spurðu þat, ok austr yfir sanda til Hofs.
Þá sendu þeir orð Starkaði undir Þríhyrningi; ok fundusk þeir þar allir er at
Gunnari skyldu fara, ok réðu hversu at skyldi fara.

\subsection*{Runic}
\fontspec{Junicode}
ᚠᛁᛋᚳ ᚠᛚᚩᛞᚢ ᚪᚻᚩᚠ ᚩᚾ ᚠᛖᚱᚷᛖᚾᛒᛖᚱᛁᚷ ᚹᚪᚱᚦ ᚷᚪ᛬ᛇᚱᛁᚳ ᚷᚱᚩᚱᚾ ᚦᚨᚱ ᚻᛖ ᚩᚾ ᚷᚱᛖᚢᛏ ᚷᛁᛇᚹᚩᛗ
ᚻᚱᚩᚾᚨᛇ ᛒᚪᚾ\\
ᚱᚩᛗᚹᚪᛚᚢᛇ ᚪᚾᛞ ᚱᛖᚢᛗᚹᚪᛚᚢᛇ ᛏᚹᛟᚷᛖᚾ ᚷᛁᛒᚱᚩᚦᚫᚱ ᚪᚠᛟᛞᛞᚫ ᛞᛁᚫ ᚹᚣᛚᛁᚠ ᚩᚾ ᚱᚩᛗᚫ\linebreak[0]ᚳᚫᛇᛏᛁ᛬
ᚩᚦᛚᚫ ᚢᚾᚾᛖᚷ

\subsection*{Latin}

{\small\itshape Junicode contains the most common Latin abbreviations,
  making it suitable for diplomatic editions of Latin texts.}\\[1ex]
{\addfontfeatures{StyleSet=altogonek}Adiuuanos dſ̄ ſalutariſ noſter \&
 ꝓpt̄ głam nominiſ tui dnē liƀanoſ· \& ꝓpitiuſ eſto peccatiſ noſtriſ
 ꝓpter nomen tuum· Ne forte dicant ingentib: ubi eſt dſ̄ eorum \&
  innoteſcat innationib: corā oculiſ nr̄iſ· Poſuerunt moſticina
  ſeruorū ruorū eſcaſ uolatilib: cęli carneſ ſcōꝝ tuoꝝ beſtiiſ tenice·
  Facti ſum obꝓbrium uiciniſ nr̄iſ·}

\subsection*{Gothic Transliteration}

jabai auk ƕas gasaiƕiþ þuk þana habandan kunþi in galiuge stada
anakumbjandan, niu miþwissei is siukis wisandins timrjada du
galiugagudam gasaliþ matjan?  fraqistniþ auk sa unmahteiga ana
þeinamma witubnja broþar in þize Xristus gaswalt.  swaþ~þan
frawaurkjandans wiþra broþruns, slahandans ize gahugd siuka, du
Xristau frawaurkeiþ.

\subsection*{Sanskrit Transliteration}

\noindent mānaṃ dvividhaṃ viṣayadvai vidyātśaktyaśaktitaḥ \\
     arthakriyāyāṃ keśadirnārtho ’narthādhimokṣataḥ\\[1ex]
sadṛśāsadṛśatvācca viṣayāviṣayatvataḥ \\
     śabdasyānyanimittānāṃ bhāve dhīsadasattvataḥ

\subsection*{International Phonetic Alphabet}
\fontspec[IPAMode=on]{Junicode}
hwɑn θɑt ɑːprɪl wiθ is ʃuːrəs soːtə θə drʊxt ɔf mɑrʧ hɑθ peːrsəd toː
θə roːte ɑnd bɑːðəd ɛvrɪ væɪn ɪn swɪʧ lɪkuːr ɔf hwɪʧ vɛrtɪu
ɛnʤɛndrəd ɪs θə fluːr hwɑn zɛfɪrʊs eːk wɪθ hɪs sweːtə bræːθ
\fontspec{Junicode}

\subsection*{Greek}

{\small\itshape The Greek typeface (available only in the regular
  face) is based on the Greek Double Pica cut by Alexander Wilson of
  Glasgow in the eighteenth century. It is not really suitable for
  setting modern Greek; those who want a more modern Greek face that
  harmonizes well with Junicode should consider GFS Didot
  Classic or GFS Porson.}\\[1ex]
βίβλος
γενέσεως ἰησοῦ χριστοῦ υἱοῦ δαυὶδ
υἱοῦ ἀβραάμ.
ἀβραὰμ
ἐγέννησεν τὸν ἰσαάκ, ἰσαὰκ δὲ ἐγέννησεν
τὸν ἰακώβ, ἰακὼβ δὲ ἐγέννησεν τὸν
ἰούδαν καὶ τοὺς ἀδελφοὺς αὐτοῦ,
ἰούδας
δὲ ἐγέννησεν τὸν φάρες καὶ τὸν ζάρα
ἐκ τῆς θαμάρ, φάρες δὲ ἐγέννησεν τὸν
ἑσρώμ, ἑσρὼμ δὲ ἐγέννησεν τὸν ἀράμ,
ἀρὰμ
δὲ ἐγέννησεν τὸν ἀμιναδάβ, ἀμιναδὰβ
δὲ ἐγέννησεν τὸν ναασσών, ναασσὼν δὲ
ἐγέννησεν τὸν σαλμών,
σαλμὼν
δὲ ἐγέννησεν τὸν βόες ἐκ τῆς ῥαχάβ,
βόες δὲ ἐγέννησεν

\subsection*{Lithuanian}

{\small\itshape Lithuanian poses several typographical challenges. An
  accented i retains its dot: i̇́; and certain characters with ogonek
  must avoid colliding with a following j:
  {\upshape\addfontfeatures{Contextual=on} ęj ųj}. Make sure
  Contextual Alternates (calt) is turned on; for i̇́, use i followed
  by non-spacing dot accent (0307) and acute (0301).}\\[1ex]
Visa žemė turėjo vieną kalbą ir tuos pačius žodžius.  Kai žmonės
kėlėsi iš rytų, jie rado slėnį Šinaro krašte ir ten įsikūrė.  Vieni
kitiems sakė: Eime, pasidirbkime plytų ir jas išdekime. – Vietoj
akmens jie naudojo plytas, o vietoj kalkių – bitumą.  Eime, – jie
sakė, – pasistatykime miestą ir bokštą su dangų siekiančia viršūne ir
pasidarykime sau vardą, kad nebūtume išblaškyti po visą žemės veidą.

\subsection*{Polish}
{\small\itshape At the urging of Polish type designer Adam Twardoch,
the default shape and position of ogonek in Junicode are now suitable
for modern Polish. Suggestions for further improvements are solicited.}\\[1ex]
Mieszkańcy całej ziemi mieli jedną mowę, czyli jednakowe słowa.  A
gdy wędrowali ze wschodu, napotkali równinę w kraju Szinear i tam
zamieszkali.  I mówili jeden do drugiego: Chodźcie, wyrabiajmy cegłę
i wypalmy ją w ogniu. A gdy już mieli cegłę zamiast kamieni i smołę
zamiast zaprawy murarskiej, rzekli: Chodźcie, zbudujemy sobie miasto
i wieżę, której wierzchołek będzie sięgał nieba, i w ten sposób
uczynimy sobie znak, abyśmy się nie rozproszyli po całej ziemi.

\pagebreak

\fontspec{Junicode}
\noindent {\tiny \sampletext} {\small \sampletext} {\large \sampletext}
{\Large \sampletext} {\LARGE \sampletext} {\huge \sampletext}\\

{\itshape\noindent {\tiny \sampletext} {\small \sampletext} {\large \sampletext}
{\Large \sampletext} {\LARGE \sampletext} {\huge \sampletext}}\\

\noindent {\scshape {\tiny \sctext} {\small \sctext} {\large \sctext}
{\Large \sctext} {\LARGE \sctext}}

\pagebreak

{\bfseries\noindent {\tiny \sampletext} {\small \sampletext} {\large \sampletext}
{\Large \sampletext} {\LARGE \sampletext} {\huge \sampletext}}\\

{\bfseries\itshape\noindent {\tiny \sampletext} {\small \sampletext} {\large \sampletext}
{\Large \sampletext} {\LARGE \sampletext} {\huge \sampletext}}\\

\noindent{\LARGE abcdefghijklmnopqrstuvwxyz æðþȝ\\
ABCDEFGHIJKLMNOPQRSTUVWXYZ ÆÐÞȜ\\
αβγδεζηθικλμνξοπρςστυφχψω\\
ΑΒΓΔΕΖΗΘΙΚΛΜΝΞΟΠΡΣΤΥΦΧΨΩ}

\pagebreak


\noindent The Junicode font is available at
http://junicode.sourceforge.net/. You can also find it in the
repositories of many Linux distributions, and also via CTAN. Visit the
Junicode Project Page at SourceForge to leave feature requests and bug
reports. Contributions are welcome: if you wish to contribute to
Junicode, leave a patch at the Project Page or contact the
developer.\\

\subsection*{Developer}
Peter S. Baker, University of Virginia

\subsection*{Contributors}
Denis Moyogo Jacquerye\\
Adam Buchbinder\\
Pablo Rodriguez\\

\noindent Thanks to the many users who have submitted feature requests
and bug reports.\\

\def\reflect#1{{\setbox0=\hbox{#1}\rlap{\kern0.5\wd0
  \special{x:gsave}\special{x:scale -1 1}}\box0 \special{x:grestore}}}
\def\XeTeX{\leavevmode
  \setbox0=\hbox{X\lower.5ex\hbox{\kern-.15em\reflect{E}}\kern-.1667em \TeX}%
  \dp0=0pt\ht0=0pt\box0 }

\noindent This document was set with {\XeTeX}.
\end{document}
